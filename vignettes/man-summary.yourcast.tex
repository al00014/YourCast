\HeaderA{summary.yourcast}{Summarize yourcast output}{summary.yourcast}
%
\begin{Description}\relax
yourcast class summary function
\end{Description}
%
\begin{Usage}
\begin{verbatim}
## S3 method for class 'yourcast'
summary(object, ...)
\end{verbatim}
\end{Usage}
%
\begin{Arguments}
\begin{ldescription}
\item[\code{object}] \code{\LinkA{yourcast}{yourcast}} output object to be summarized
\item[\code{...}] Arguments to be passed to or from other methods.
\end{ldescription}
\end{Arguments}
%
\begin{Value}
A list of summary objects of class `summary.yourcast':

\begin{ldescription}
\item[\code{sample.frame}] Four element vector containing, in order,
the start and end time periods to be used for the observed
data and the start and end time periods to be forecast

\item[\code{params}] Vector. Smoothing parameters used in model.

\item[\code{model}] String. Estimation technique used.

\item[\code{formula}] Language. Formula used in analysis.

\item[\code{numcs}] Integer. Number of cross sections used in analysis.

\item[\code{cntry.codes}] Vector. Unique geographical codes of cross
sections used in analysis.

\item[\code{cntry.names}] Vector. If \code{G.names} dataframe submitted
to \code{\LinkA{yourcast}{yourcast}}, the names of all unique geographical
codes of cross sections used in analysis.

\item[\code{coef}] List. List of matrices with estimated coefficents for
each cross section. For Lee-Carter models, only one set of
coefficents calculated. Standard errors are not calculated for
these coefficients.
\end{ldescription}
\end{Value}
%
\begin{Author}\relax
Jon Bischof \email{jbischof@fas.harvard.edu}
\end{Author}
%
\begin{References}\relax
\url{http://gking.harvard.edu/yourcast}
\end{References}
%
\begin{SeeAlso}\relax
\code{\LinkA{yourcast}{yourcast}},\code{\LinkA{print.summary.yourcast}{print.summary.yourcast}} 
\end{SeeAlso}
