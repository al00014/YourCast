\HeaderA{yourprep}{Data object creation wizard for YourCast}{yourprep}
%
\begin{Description}\relax
Builds the data object for \code{\LinkA{yourcast}{yourcast}}
function from files in working directory or other specified
directory and checks for errors
\end{Description}
%
\begin{Usage}
\begin{verbatim}
yourprep(dpath=getwd(),tag="csid",index.code="ggggaa",
                     datalist=NULL,G.names=NULL,A.names=NULL,
                     T.names=NULL,proximity=NULL,year.var=FALSE,
                     sample.frame=NULL,summary=FALSE,verbose=FALSE,

                     #lagging utility
                     lag=NULL,formula=NULL,vars.nolag=NULL)
\end{verbatim}
\end{Usage}
%
\begin{Arguments}
\begin{ldescription}

\item[\code{dpath}] String. Name of the directory where data files are
stored.  If \code{NULL} then defaults to working directory.
Default: \code{NULL}

\item[\code{tag}] String. Group of characters placed before CSID code in
filenames to indicate which files in \code{dpath} function
should load. The \code{tag} can also be used to differentiate
between different groups to be considered in separate
analysis; for example, `m' for male deaths and
`f' for female deaths. Default: \code{"csid"}

\item[\code{index.code}] String indicating how the CSID index variable is
coded in the input data. Between 0 and 4 of the following two
characters are used in this order: \code{g} for the geographic
index (such
as country) and \code{a} for a grouped continuous variable
like
an age group. For example, \code{ggggaa} would have the
function interpret
`245045' by using `2450' as the country code and
`45' as the age group. Default: \code{"ggggaa"}



\item[\code{datalist}] A list of cross section dataframes already loaded into
the workspace to be added to the \code{dataobj}. Names of
list elements should be the numerical CSID code for each
cross section, and dataframes should be formated identically
to files loaded from an external directory (see Details)

\item[\code{A.names, G.names, T.names}] String. Filename of optional
two-column data files that list all valid numerical codes
(in the first column) and corresponding alphanumeric names
(optionally in the second column) for the indices
corresponding to geographic areas in \code{G.names}, age
groups in \code{A.names}, and time periods in
\code{T.names}. Function will search \code{dpath} for file
with specified name; please include column labels. The
optional alphanumeric identifiers are most commonly only
used for geographic areas since numerical values for age
groups and time periods are usually meaingful on their
own. However, if other grouped continuous variable used in
place of ages, for example, specifying these labels will be
important for output to be meaningful. NOTE: Auxiliary files
will loaded automatically by \code{yourprep()} if they are
saved in the \code{dpath} and labeled with the \code{tag}
specified by
the user. See `Details' section for more
infromation. Default: \code{NULL}


\item[\code{proximity}] Data file with codes to construct the symmetric
matrix (geographic region by geographic region) of proximity
scores for geographic smoothing used by the `map' and `bayes'
methods. The larger the relative score, the more proximate
that pair of countries is in the prior; a zero element means
the two geographic areas are unrelated (the diagonal is
ignored).  Each row of the \code{proximity} file has three
columns, consisting of geographic codes for two countries
and a score indicating the proximity or similarity of the
two geographic regions; please include column labels. For
convenience, geographic regions that are unrelated (and
would have zero entries in the symmetric matrix) may be
omitted from \code{proximity}. In addition, \code{proximity}
may include rows corresponding to geographic regions not
included in the present analysis. Default: \code{NULL}

\item[\code{year.var}] Boolean. Should be \code{TRUE} if \code{year} coded as
separate variable rather than as rowname for cross section
data files.  Function will look for \code{year} variable to
use as rownames and then drop it from the dataframe. Change
will only be made to dataframe if it does not already have
rownames or if exisiting rownames are merely a
`1...N' index of row numbers, so it is possible to
apply correction even if some cross sections do not have a
\code{year} variable and already have the correct
rownames. Default: \code{FALSE}

\item[\code{sample.frame}] Optional four element vector containing, in order,
the start and end time periods to be used for the observed
data and the start and end time periods to be forecast. All
cross sections do not have to begin at starting date, but
must contain all years after the first observed
value. Variables to be forecasted should be coded as
\code{NA} in the out-of-sample period. Note that this makes
it easy to reserve a range of values of the dependent
variable for out-of-sample forecasting evaluation; our
\code{summary} and \code{plot} functions in
\code{\LinkA{yourcast}{yourcast}} will make these comparisons
automatically if the out-of-sample data are
included. \code{yourprep()} uses this information only to
verify that cross sections are correctly
constructed, but it should also be included if one wants to
use the lag utility. Default: \code{NULL}

\item[\code{summary}] Boolean. If \code{TRUE}, means for available
observations on each variable are displayed for the cross
sections read by \code{yourprep()}. Default: \code{FALSE}

\item[\code{verbose}] Boolean. If \code{TRUE}, function prints name of each
cross section or auxiliary file as it is read into the
\code{dataobj}. Default: \code{FALSE}

\item[\code{lag}] Number of years covariate data needs to be lagged from
current position is cross section files. See `Details' for more
information. Default: \code{NULL}

\item[\code{formula}] Formula. The formula that one will use in the subsequent run of
\code{yourcast()}. This helps the lagging utility distinguish between
the response variable (which will not be shifted between cross
sections) and the covariates of interest that should be lagged and
included in the final cross sections of the dataobj. If the covariate
`index' is included in the formula, the lagging utility will
include a variable in the cross sections that starts from 1 and counts
the number of time periods since the start of the cross section.
If a lag is requested, the formula argument must be specified.
Default: \code{NULL}

\item[\code{vars.nolag}] Vector of strings. Vector of variables to be included
in the dataobj but not lagged. These variables do not need to be
included in the formula, and if found there will not ignored when the
other covariates are lagged.


\end{ldescription}
\end{Arguments}
%
\begin{Details}\relax
Creates \code{dataobj} input for \code{\LinkA{yourcast}{yourcast}} from
files in working directory or other specified directory. Checks
that all cross sections in \code{data} list titled properly and
if all years up to last predicted year included in the dataframes
(if \code{sample.frame} argument specified). Please note, however,
that all cross sections from the same geographic area must have the
same observation and prediction years in the dataframe (even if
\code{NA}) for the graphing software \code{\LinkA{plot.yourcast}{plot.yourcast}} to
work.

The cross section files must be named according to the CSID
identifiers for country code and age group, preceeded by the
specified tag (default: \code{"csid"}) so that \code{yourprep()} can
identify the file from other files in the dpath. For example, for
the USA (country code 2450) time series of 45 year old
individuals, the file name should be \file{csid245045.txt} if the
tag is left as the default. Files must have an extension so that
the program can recognize how the data is coded. Currently, fixed
width text files (\file{*.txt}), comma-separated values
(\file{*.csv}), and Stata v.5-10 (\file{*.dta}) files are
supported, and multiple file types may be used in the same run of
the program. \file{*.Rdata} objects can be included with the
\code{datalist} option after they are loaded to a list in the
workspace. \code{yourprep()} includes diagnostics to ensure that
objects are properly named and not included accidentally, but
users should examine the specified \code{dpath} before running
\code{yourprep()} to minimize errors.

Each cross section file should be labeled columns of time-series
data for the dependent variable(s) (e.g., disease, pop) and the
covariates that will be used in the forecast. The rownames for
the dataframe should be the observation year (if the year is
coded as a separate variable, set \code{year.var=TRUE}). The
files must contain the full time series that will be specified in
the \code{sample.frame} argument in \code{\LinkA{yourcast}{yourcast}} after
the first observed year. For instance, if
\code{sample.frame=c(1950,2000,2001,2030)}, then files would have
observations that start between 1950 and 2000 and include all
other years (even if the entries are \code{NA}) up to the last
year of prediction, i.e., 2030.

Optional auxiliary files such as \code{G.names} should be named
according to the filename specified in the respective
arguments. If specified, these files must have extensions and be
coded in one of the three supported file types. However, these
files will be automatically loaded by \code{yourprep()} if they are
saved in the \code{dpath} and labeled with the tag specified by the
user. The default names for these files must be used (e.g.,
`G.names' and `proximity'). For example,
if the \code{tag} is left as the default and there is a file in the
\code{dpath} labeled \file{csid.G.names.txt}, \code{yourprep()}
will load this
automatically and save the input as the \code{G.names} element of
the `dataobj' list. \code{yourprep()} arguments such as
\code{G.names} take precedence over \file{TAG.*} files in
the\code{dpath}.

\code{yourprep()} also includes a lagging utility (activated once
one specifies a lag length with the `lag' argument). This
utility is useful for when the data in each cross section is, for
example, the response and covariates for 50 year olds in each year
but the desired content for each cross section is the response for
50 year olds and the covariates for 25 year olds 25 years prior to
each year (implying a lag of 25 years). In order to have
\code{yourprep()} perform this lagging automatically, include cross
sections for each age group with data starting the same number of
years before the first observation year as the requested lag
period. Thus if \code{lag=25} and the first observation year is
1950, then the cross sections should all start at 1925. Age groups
younger than the length of the lag will not retain covariate data
(except perhaps an `index' variable) in the output
object. The covariates lagged are the predictor variables specified
in the formula argument.

If data for a cohort 25 years (in this case) younger is not
available for some cohort over age 25, \code{yourprep()} will look
for the closest cohort available and issue a warning message. 

\end{Details}
%
\begin{Value}
\begin{ldescription}
\item[\code{dataobj}] A list with several components:

\begin{description}
 

\item[data] A list with the cross-sectional data matrices as elements.

\item[proximity] A three-column matrix of proximity scores for geographic smoothing used by the `map'
and `bayes' methods. For each row, the first two columns indicate the country pair. The third column indicates the proximity score. The larger each score, the more
proximate that pair of countries is in the prior; a zero element means
the two geographic areas are unrelated (the diagonal is ignored).

\item[G.names, A.names, T.names] Optional two-column dataframes that
list all valid numerical codes (in the first column, labeled codes)
and corresponding alphanumeric names (optionally in the second
column, labeled name) for the indices corresponding to the geographic
areas in \code{G.names}, age groups in \code{A.names}, and time
periods in \code{T.names}.

\item[index.code] A string indicating how the index variable is coded
in the input data.
\end{description}
\end{ldescription}
\end{Value}
%
\begin{Author}\relax
Jon Bischof \email{jbischof@fas.harvard.edu}
\end{Author}
%
\begin{References}\relax
\url{http://gking.harvard.edu/yourcast}
\end{References}
%
\begin{SeeAlso}\relax
\code{\LinkA{yourcast}{yourcast}} function and documentation
(\code{help(yourcast)})
\end{SeeAlso}
%
\begin{Examples}
\begin{ExampleCode}
## Not run: 
# Working directory automatically set to directory with cross
# section and auxiliary files to begin. Files for this example
# in 'data' folder of YourCast library.

#Old working directory to be restored later
oldwd <- getwd()
# Now setting wd to 'data' folder in YourCast library
setwd(system.file("data",package="YourCast"))

# Simple run of the function, using option that turns year variable
# into label in each cs. Use sample.frame argument for all diagnostics
# to work
 
dta <- yourprep(G.names="cntry.codes.txt", proximity="proximity.txt",
year.var=TRUE,verbose=TRUE,sample.frame=c(1950,2000,2001,2030))


# With summary output (means of variables in each cross section) 


dta <- yourprep(G.names="cntry.codes.txt", proximity ="proximity.txt",
year.var=TRUE,summary=TRUE)


# Function can also add datafiles already loaded into R as objects in
# the workspace with "datalist" option if put into a list and properly
# labeled. All diagnostics still performed 
# 'csid204545', etc., are dataframes in workspace

# Labels changed to nonsense ones so as not to confuse with other files

data(csid204545)
data(csid204550)
data(csid204555)

datalist <- list("123456"=csid204545,"234567"=csid204550,
"345678"=csid204555) 

# Verbose option turned on and datalist argument added 

dta <- yourprep(G.names="cntry.codes.txt", proximity="proximity.txt",
year.var=TRUE,verbose=TRUE,datalist=datalist)

# Setting working directory back
setwd(oldwd)
rm(oldwd)

## End(Not run)
\end{ExampleCode}
\end{Examples}
